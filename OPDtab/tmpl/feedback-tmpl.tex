\documentclass{scrartcl}
\usepackage[utf8]{inputenc}
\usepackage[T1]{fontenc}
\usepackage[ngerman]{babel}
\usepackage[a4paper, top=3cm, bottom=1cm, left=2cm, right=2cm]{geometry}
\usepackage{latexsym}
\usepackage{array}
\usepackage{fancyhdr}
\pagestyle{fancy}

\newcommand{\questionsix}[3]{%
#3%

\medskip%

{%
  \centering%
  \begin{tabular}{>{\centering}p{0.13\textwidth}%
                  >{\centering}p{0.13\textwidth}%
                  >{\centering}p{0.13\textwidth}%
                  >{\centering}p{0.13\textwidth}%
                  >{\centering}p{0.13\textwidth}%
                  >{\centering\arraybackslash}p{0.13\textwidth}}%
    $\Box$ & $\Box$ & $\Box$ & $\Box$ & $\Box$ & $\Box$ \\%
    #1 & & & & & #2%
  \end{tabular}%
}%
}

\newcommand{\questionseven}[4]{%
#4%

\medskip%

{%
  \centering%
  \begin{tabular}{>{\centering}p{0.11\textwidth}%
                  >{\centering}p{0.11\textwidth}%
                  >{\centering}p{0.11\textwidth}%
                  >{\centering}p{0.11\textwidth}%
                  >{\centering}p{0.11\textwidth}%
                  >{\centering}p{0.11\textwidth}%
                  >{\centering\arraybackslash}p{0.11\textwidth}}%
    $\Box$ & $\Box$ & $\Box$ & $\Box$ & $\Box$ & $\Box$ & $\Box$ \\%
    #1 & & & #2 & & & #3%
  \end{tabular}%
}%
}

\newcommand{\questionfour}[1]{%
#1%

\medskip%

{%
  \centering%
  \begin{tabular}{>{\centering}p{0.23\textwidth}%
                  >{\centering}p{0.23\textwidth}%
                  >{\centering}p{0.23\textwidth}%
                  >{\centering\arraybackslash}p{0.23\textwidth}}%
    $\Box$ & $\Box$ & $\Box$ & $\Box$ \\%
    voll und ganz & eher ja & eher nicht & gar nicht %
  \end{tabular}%
}%
}

\newcommand{\strengthsandweaknesses}[1]{%
{%
  \centering%
  \begin{tabular}{|p{0.45\textwidth}|p{0.45\textwidth}|}%
    \hline%
    Was waren aus Deiner Sicht die Stärken des/der Juror*in? &%
    Was waren aus Deiner Sicht die Schwächen des/der Juror*in? \\%
    \hline%
    \bigskip & \\[#1]%
    \hline%
  \end{tabular}%
}%
}

\newcommand{\feedbackfromfor}[2]{%
\begin{tabular}{ll}%
  Feedback \textbf{von}: & #1 \\%
  Feedback \textbf{für}: & \textbf{#2}%
\end{tabular}%
}

\begin{document}

\setlength{\parindent}{0pt}

<!-BEGIN:ROOMS->

  \fancyhf{}
  \lhead{Raum [ROOM]}
  \rhead{Runde [ROUND]}

  <!-BEGIN:TEAMSFORCHAIR->
    \section*{Teams für Hauptjuror*innen}

    \feedbackfromfor{[TEAMNAME] (Team)}{[CHAIRNAME] (HJ)}

    \bigskip

    Liebe Redner*innen,

    \medskip

    alle Juror*innen, insbesondere aber das Chefjurorenteam, stehen euch
    jederzeit für Lob, Kritik und Anregungen zur Verfügung. Zusätzlich dient
    dieser Bogen dazu, Kritik und Lob zu den Leistungen der Juror*innen
    auszusprechen oder weitere Anregungen zu geben. Dieser Bogen ist ein
    wichtiges Instrument, um eine gerechte und kompetente Bewertung der
    Debatten zu gewährleisten. Eure Meinung ist uns wichtig, also macht bitte
    Gebrauch von dieser Möglichkeit. Du kannst gerne die Rückseite für weitere
    Anmerkungen benutzen!

    \medskip
    \hrulefill
    \medskip

    \questionseven{viel zu gut}{genau richtig}{viel zu schlecht}{
      Die Bewertung/Bepunktung meiner Leistung fand ich\dots
    }
    \medskip

    \questionseven{viel zu gut}{genau richtig}{viel zu schlecht}{
      Die Bewertung/Bepunktung der anderen RednerInnen war\dots
    }
    \medskip

    \questionsix{sehr gut}{sehr schlecht}{
      Das Feedback der Jurorin/des Jurors war aus meiner Sicht\dots
    }
    \medskip

    \questionfour{
      Traust du dem/der Juror*in zu, auch in der nächsten Runde wieder das
      Feedback zu geben?
    }
    \medskip

    \strengthsandweaknesses{6cm}

    \newpage
  <!-END:TEAMSFORCHAIR->

  <!-BEGIN:FREEFORCHAIR->
    \section*{Freie Redner*innen für Hauptjuror*innen}

    \feedbackfromfor{[FREENAME] (FR)}{[CHAIRNAME] (HJ)}

    \bigskip

    Liebe Redner*innen,

    \medskip

    alle Juror*innen, insbesondere aber das Chefjurorenteam, stehen euch
    jederzeit für Lob, Kritik und Anregungen zur Verfügung. Zusätzlich dient
    dieser Bogen dazu, Kritik und Lob zu den Leistungen der Juror*innen
    auszusprechen oder weitere Anregungen zu geben. Dieser Bogen ist ein
    wichtiges Instrument, um eine gerechte und kompetente Bewertung der
    Debatten zu gewährleisten. Eure Meinung ist uns wichtig, also macht bitte
    Gebrauch von dieser Möglichkeit. Du kannst gerne die Rückseite für weitere
    Anmerkungen benutzen!

    \medskip
    \hrulefill
    \medskip

    \questionseven{viel zu gut}{genau richtig}{viel zu schlecht}{
      Die Bewertung/Bepunktung meiner Leistung fand ich\dots
    }
    \medskip

    \questionseven{viel zu gut}{genau richtig}{viel zu schlecht}{
      Die Bewertung/Bepunktung der anderen RednerInnen war\dots
    }
    \medskip

    \questionsix{sehr gut}{sehr schlecht}{
      Das Feedback der Jurorin/des Jurors war aus meiner Sicht\dots
    }
    \medskip

    \questionfour{
      Traust du dem/der Juror*in zu, auch in der nächsten Runde wieder das
      Feedback zu geben?
    }
    \medskip

    \strengthsandweaknesses{6cm}

    \newpage
  <!-END:FREEFORCHAIR->

  <!-BEGIN:CHAIRFORPANELIST->
    \section*{Feedback von Hauptjuror*innen für Nebenjuror*innen}

    \feedbackfromfor{[CHAIRNAME] (HJ)}{[PANELISTNAME] (NJ)}

    \bigskip

    Liebe Hauptjuror*innen,

    \medskip

    wir, das Chefjurorenteam, stehen euch jederzeit für
    Lob, Kritik und Anregungen zur Verfügung. Zugleich sind wir auf eure
    Mithilfe angewiesen, um eine gerechte und kompetente Bewertung aller
    Debatten zu gewährleisten. Deshalb dient dieser Bogen dazu, Kritik und Lob
    zu den Leistungen eurer Mitjuror*innen auszusprechen oder weitere
    Anregungen zu geben. Eure Meinung ist uns wichtig, also macht bitte
    Gebrauch von dieser Möglichkeit. Auch die Rückseite steht Euch für weitere
    Anmerkungen zur Verfügung.

    \medskip
    \hrulefill
    \medskip

    \questionsix{völlig}{gar nicht}{
      Die Bewertungen der Rednerleistungen durch den/die Juror*in waren für
      mich nachvollziehbar.
    }
    \medskip

    \questionsix{völlig}{gar nicht}{
      Der/die Juror*in hat sich konstruktiv in die Jurierdiskussion
      eingebracht.
    }
    \medskip

    \questionfour{
      Traust du dem/der Juror*in zu, in der nächsten Runde Feedback zu geben?
    }
    \medskip

    \strengthsandweaknesses{9cm}

    \newpage
  <!-END:CHAIRFORPANELIST->

  <!-BEGIN:PANELISTFORJUDGE->
    \section*{Feedback von Nebenjuror*innen für Mitjuror*innen}

    \feedbackfromfor{[PANELISTNAME] (NJ)}{[JUDGENAME]}

    \bigskip

    Liebe Nebenjuror*innen,

    \medskip

    wir, das Chefjurorenteam, stehen euch jederzeit für
    Lob, Kritik und Anregungen zur Verfügung. Zugleich sind wir auf eure
    Mithilfe angewiesen, um eine gerechte und kompetente Bewertung aller
    Debatten zu gewährleisten. Deshalb dient dieser Bogen dazu, Kritik und Lob
    zu den Leistungen eurer Mitjuror*innen auszusprechen oder weitere
    Anregungen zu geben. Eure Meinung ist uns wichtig, also macht bitte
    Gebrauch von dieser Möglichkeit. Auch die Rückseite steht Euch für weitere
    Anmerkungen zur Verfügung.

    \medskip
    \hrulefill
    \medskip

    \questionsix{völlig}{gar nicht}{
      Die Bewertungen der Rednerleistungen durch den/die Juror*in waren für
      mich nachvollziehbar.
    }
    \medskip

    \questionsix{völlig}{gar nicht}{
      Der/die Juror*in hat die Jurierdiskussion konstruktiv moderiert.
    }
    \medskip

    \questionsix{sehr gut}{sehr schlecht}{
      Das Feedback der Jurorin/des Jurors war aus meiner Sicht\dots
    }
    \medskip

    \questionfour{
      Traust du dem/der Juror*in zu, auch in der nächsten Runde wieder das
      Feedback zu geben?
    }
    \medskip

    \strengthsandweaknesses{8cm}

    \newpage
  <!-END:PANELISTFORJUDGE->
<!-END:ROOMS->
\end{document}
